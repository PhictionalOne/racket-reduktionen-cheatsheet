%! TEX root = 'main.tex'

%Template 
\documentclass[a4paper]{scrartcl}

\usepackage{geometry,forloop,fancyhdr,fancybox,lastpage}
\usepackage{enumitem}
\usepackage{adjustbox}
\geometry{a4paper,left=1cm, right=1cm, top=3cm, bottom=1cm}

% Fonts
\usepackage{fontspec}
% \usepackage[]{ucs}   % use φ's intext
\usepackage[ngerman]{babel}

\usepackage[sfdefault,scaled=.85]{FiraSans}     % Sans font
\usepackage{newtxsf}
\renewcommand*\oldstylenums[1]{{\firaoldstyle #1}}

\setmonofont{Fira Code}[Scale=0.8,Contextuals=Alternate]  % Code Font
\makeatletter
\renewcommand*\verbatim@nolig@list{}
\makeatother

% Code
\usepackage[dvipsnames]{xcolor}
\usepackage[outputdir=build]{minted}
\usemintedstyle{trac}

% todonotes
\usepackage{todonotes}

%Math
\usepackage{amsmath,amssymb,tabularx}
\usepackage{ stmaryrd }

%Figures
\usepackage{graphicx,tikz,color,float}
\usetikzlibrary{shapes,trees}

\usepackage{hyperref}
\usepackage{pdfpages}

%Algorithms
\usepackage[ruled,linesnumbered]{algorithm2e}

% TColorbox
\usepackage{tcolorbox}

% Colors
\definecolor{codegreen}{rgb}{0,0.6,0}
\definecolor{codegray}{rgb}{0.5,0.5,0.5}
\definecolor{codepurple}{rgb}{0.58,0,0.82}
\definecolor{backcolour}{rgb}{0.95,0.95,0.92}
\definecolor{codegreen-task}{rgb}{0,0.3,0}
\definecolor{codepurple-task}{rgb}{0.38,0,0.62}
\definecolor{backcolour-task}{rgb}{0.95,0.95,0.95}

\definecolor{lightolive}{RGB}{216,202,157}
\definecolor{eigengrau}{RGB}{22,22,29}


%Kopf- und Fußzeile
\pagestyle {fancy}
\fancyhead[L]{\TUTOR}
\fancyhead[C]{\COURSE}
\fancyhead[R]{\today}

\fancyfoot[L]{}
\fancyfoot[C]{}
\fancyfoot[R]{Seite \thepage}

%Formatierung der Überschrift, hier nichts ändern
\def\header#1#2{
	\begin{center}
		{\Large\textbf{Übungsblatt #1}}\\
		{(Abgabetermin #2)}
	\end{center}
}

%Definition der Punktetabelle, hier nichts ändern
\newcounter{punktelistectr}
\newcounter{punkte}
\newcommand{\punkteliste}[2]{%
	\setcounter{punkte}{#2}%
	\addtocounter{punkte}{-#1}%
	\stepcounter{punkte}%<-- also punkte = m-n+1 = Anzahl Spalten[1]
	\begin{center}%
		\begin{tabularx}{\linewidth}[]{@{}*{\thepunkte}{>{\centering\arraybackslash} X|}@{}>{\centering\arraybackslash}X}
			\forloop{punktelistectr}{#1}{\value{punktelistectr} < #2 } %
			{%
			\thepunktelistectr &
			}
			#2                 & $\Sigma$ \\
			\hline
			\forloop{punktelistectr}{#1}{\value{punktelistectr} < #2 } %
			{%
			                   &
			}                  &          \\
			\forloop{punktelistectr}{#1}{\value{punktelistectr} < #2 } %
			{%
			                   &
			}                  &          \\
		\end{tabularx}
	\end{center}
}

\newcommand{\aufgabe}[2]{
	\section*{Aufgabe #1\hfill\small\textcolor{gray}{(#2 Punkte)}}
}

% Math notation
\newcommand{\bigo}{\mathcal{O}}
\newcommand{\qed}{\begin{flushright}$\square$\end{flushright}}
\newcommand{\kstn}{\qed}

\newcommand{\bbN}{\mathbb{N}}
\newcommand{\bbZ}{\mathbb{Z}}
\newcommand{\bbQ}{\mathbb{Q}}
\newcommand{\bbR}{\mathbb{R}}
\newcommand{\calO}{\mathcal{O}}
\newcommand{\bfE}{\mathbf{E}}
\newcommand{\bfP}{\mathbf{P}}
\DeclareMathOperator{\var}{Var}
\DeclareMathOperator*{\argmax}{arg\,max}
\DeclareMathOperator*{\argmin}{arg\,min}

% todonotes
\newcommand{\phiinline}[1]{
	\todo[inline, author=Phi,
		backgroundcolor=lightolive, textcolor=eigengrau]{#1}
}
\newcommand{\phitodo}[1]{
	\todo[author=Phi, linecolor=lightolive,
		backgroundcolor=lightolive, textcolor=eigengrau]{#1}
}

% tcolorbox
\tcbuselibrary{breakable}

\newtcolorbox{contentbox}[1]{
	boxrule=-0.2mm, leftrule=2mm,  middle=2mm, sharp corners,
	colbacktitle=black!5!white, coltitle=black!55!white,
	toptitle=1mm, title={\MakeUppercase{\textbf{#1}}}
}

\newtcolorbox{solutionbox}{
	boxrule=-0.2mm, leftrule=2mm,  middle=2mm, sharp corners,
	colbacktitle=black!5!white, coltitle=black!55!white, breakable,
	toptitle=1mm, title={\MakeUppercase{\textbf{Lösung}}}
}

\newtcolorbox{warningbox}[1]{
	boxrule=-0.2mm, leftrule=2mm,  middle=2mm, sharp corners,
	colback=orange!5!white, colframe=orange!75!white,
	colbacktitle=orange!5!white, coltitle=black!55!white,
	toptitle=1mm, title={\MakeUppercase{\textbf{#1}}}
}

\newtcolorbox{errorbox}[1]{
	boxrule=-0.2mm, leftrule=2mm,  middle=2mm, sharp corners,
	colback=red!5!white, colframe=red!75!black,
	colbacktitle=red!5!white, coltitle=black!55!white,
	toptitle=1mm, title={\MakeUppercase{\textbf{#1}}}
}

\newtcolorbox{infobox}[1]{
	boxrule=-0.2mm, leftrule=2mm,  middle=2mm, sharp corners,
	colback=blue!5!white, colframe=blue!75!black,
	colbacktitle=blue!5!white, coltitle=black!55!white,
	toptitle=1mm, title={\MakeUppercase{\textbf{#1}}}
}


%% Listing Style
%\lstdefinestyle{task}{
%    backgroundcolor=\color{backcolour-task},   
%    commentstyle=\color{codegreen-task},
%    keywordstyle=\ttfamily\bf\color{black},
%    numberstyle=\tiny\color{codegray},
%    stringstyle=\color{codepurple-task},
%    basicstyle=\ttfamily\footnotesize,
%    breakatwhitespace=false,         
%    breaklines=true,                 
%    captionpos=b,                    
%    keepspaces=true,                 
%    numbers=none,                    
%    numbersep=5pt,                  
%    showspaces=false,                
%    showstringspaces=false,
%    showtabs=false,                  
%    tabsize=2
%}
%
%\lstdefinestyle{solution}{
%    backgroundcolor=\color{backcolour},   
%    commentstyle=\color{codegreen},
%    keywordstyle=\color{blue},
%    numberstyle=\tiny\color{codegray},
%    stringstyle=\color{codepurple},
%    basicstyle=\ttfamily\footnotesize,
%    breakatwhitespace=false,         
%    breaklines=true,                 
%    captionpos=b,                    
%    keepspaces=true,                 
%    numbers=left,                    
%    numbersep=5pt,                  
%    showspaces=false,                
%    showstringspaces=false,
%    showtabs=false,                  
%    tabsize=2
%}
%
%\lstset{style=task}
%\lstset{style=solution}
