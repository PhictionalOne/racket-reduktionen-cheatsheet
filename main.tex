% ----------------------- TODO ---------------------------
% Diese Daten müssen pro Blatt angepasst werden:
% Diese Daten müssen einmalig pro Vorlesung angepasst werden:
\newcommand{\COURSE}{Praktische Informatik 1: Reduktionsregeln}
\newcommand{\TUTOR}{Alexander \textsc{Phi.} Goetz}
% ----------------------- TODO ---------------------------

%! TEX root = 'main.tex'

%Template 
\documentclass[a4paper]{scrartcl}

\usepackage{geometry,forloop,fancyhdr,fancybox,lastpage}
\usepackage{enumitem}
\usepackage{adjustbox}
\geometry{a4paper,left=1cm, right=1cm, top=3cm, bottom=1cm}

% Fonts
\usepackage{fontspec}
% \usepackage[]{ucs}   % use φ's intext
\usepackage[ngerman]{babel}

\usepackage[sfdefault,scaled=.85]{FiraSans}     % Sans font
\usepackage{newtxsf}
\renewcommand*\oldstylenums[1]{{\firaoldstyle #1}}

\setmonofont{Fira Code}[Scale=0.8,Contextuals=Alternate]  % Code Font
\makeatletter
\renewcommand*\verbatim@nolig@list{}
\makeatother

% Code
\usepackage[dvipsnames]{xcolor}
\usepackage[outputdir=build]{minted}
\usemintedstyle{trac}

% todonotes
\usepackage{todonotes}

%Math
\usepackage{amsmath,amssymb,tabularx}
\usepackage{ stmaryrd }

%Figures
\usepackage{graphicx,tikz,color,float}
\usetikzlibrary{shapes,trees}

\usepackage{hyperref}
\usepackage{pdfpages}

%Algorithms
\usepackage[ruled,linesnumbered]{algorithm2e}

% TColorbox
\usepackage{tcolorbox}

% Colors
\definecolor{codegreen}{rgb}{0,0.6,0}
\definecolor{codegray}{rgb}{0.5,0.5,0.5}
\definecolor{codepurple}{rgb}{0.58,0,0.82}
\definecolor{backcolour}{rgb}{0.95,0.95,0.92}
\definecolor{codegreen-task}{rgb}{0,0.3,0}
\definecolor{codepurple-task}{rgb}{0.38,0,0.62}
\definecolor{backcolour-task}{rgb}{0.95,0.95,0.95}

\definecolor{lightolive}{RGB}{216,202,157}
\definecolor{eigengrau}{RGB}{22,22,29}


%Kopf- und Fußzeile
\pagestyle {fancy}
\fancyhead[L]{\TUTOR}
\fancyhead[C]{\COURSE}
\fancyhead[R]{\today}

\fancyfoot[L]{}
\fancyfoot[C]{}
\fancyfoot[R]{Seite \thepage}

%Formatierung der Überschrift, hier nichts ändern
\def\header#1#2{
	\begin{center}
		{\Large\textbf{Übungsblatt #1}}\\
		{(Abgabetermin #2)}
	\end{center}
}

%Definition der Punktetabelle, hier nichts ändern
\newcounter{punktelistectr}
\newcounter{punkte}
\newcommand{\punkteliste}[2]{%
	\setcounter{punkte}{#2}%
	\addtocounter{punkte}{-#1}%
	\stepcounter{punkte}%<-- also punkte = m-n+1 = Anzahl Spalten[1]
	\begin{center}%
		\begin{tabularx}{\linewidth}[]{@{}*{\thepunkte}{>{\centering\arraybackslash} X|}@{}>{\centering\arraybackslash}X}
			\forloop{punktelistectr}{#1}{\value{punktelistectr} < #2 } %
			{%
			\thepunktelistectr &
			}
			#2                 & $\Sigma$ \\
			\hline
			\forloop{punktelistectr}{#1}{\value{punktelistectr} < #2 } %
			{%
			                   &
			}                  &          \\
			\forloop{punktelistectr}{#1}{\value{punktelistectr} < #2 } %
			{%
			                   &
			}                  &          \\
		\end{tabularx}
	\end{center}
}

\newcommand{\aufgabe}[2]{
	\section*{Aufgabe #1\hfill\small\textcolor{gray}{(#2 Punkte)}}
}

% Math notation
\newcommand{\bigo}{\mathcal{O}}
\newcommand{\qed}{\begin{flushright}$\square$\end{flushright}}
\newcommand{\kstn}{\qed}

\newcommand{\bbN}{\mathbb{N}}
\newcommand{\bbZ}{\mathbb{Z}}
\newcommand{\bbQ}{\mathbb{Q}}
\newcommand{\bbR}{\mathbb{R}}
\newcommand{\calO}{\mathcal{O}}
\newcommand{\bfE}{\mathbf{E}}
\newcommand{\bfP}{\mathbf{P}}
\DeclareMathOperator{\var}{Var}
\DeclareMathOperator*{\argmax}{arg\,max}
\DeclareMathOperator*{\argmin}{arg\,min}

% todonotes
\newcommand{\phiinline}[1]{
	\todo[inline, author=Phi,
		backgroundcolor=lightolive, textcolor=eigengrau]{#1}
}
\newcommand{\phitodo}[1]{
	\todo[author=Phi, linecolor=lightolive,
		backgroundcolor=lightolive, textcolor=eigengrau]{#1}
}

% tcolorbox
\tcbuselibrary{breakable}

\newtcolorbox{contentbox}[1]{
	boxrule=-0.2mm, leftrule=2mm,  middle=2mm, sharp corners,
	colbacktitle=black!5!white, coltitle=black!55!white,
	toptitle=1mm, title={\MakeUppercase{\textbf{#1}}}
}

\newtcolorbox{solutionbox}{
	boxrule=-0.2mm, leftrule=2mm,  middle=2mm, sharp corners,
	colbacktitle=black!5!white, coltitle=black!55!white, breakable,
	toptitle=1mm, title={\MakeUppercase{\textbf{Lösung}}}
}

\newtcolorbox{warningbox}[1]{
	boxrule=-0.2mm, leftrule=2mm,  middle=2mm, sharp corners,
	colback=orange!5!white, colframe=orange!75!white,
	colbacktitle=orange!5!white, coltitle=black!55!white,
	toptitle=1mm, title={\MakeUppercase{\textbf{#1}}}
}

\newtcolorbox{errorbox}[1]{
	boxrule=-0.2mm, leftrule=2mm,  middle=2mm, sharp corners,
	colback=red!5!white, colframe=red!75!black,
	colbacktitle=red!5!white, coltitle=black!55!white,
	toptitle=1mm, title={\MakeUppercase{\textbf{#1}}}
}

\newtcolorbox{infobox}[1]{
	boxrule=-0.2mm, leftrule=2mm,  middle=2mm, sharp corners,
	colback=blue!5!white, colframe=blue!75!black,
	colbacktitle=blue!5!white, coltitle=black!55!white,
	toptitle=1mm, title={\MakeUppercase{\textbf{#1}}}
}


%% Listing Style
%\lstdefinestyle{task}{
%    backgroundcolor=\color{backcolour-task},   
%    commentstyle=\color{codegreen-task},
%    keywordstyle=\ttfamily\bf\color{black},
%    numberstyle=\tiny\color{codegray},
%    stringstyle=\color{codepurple-task},
%    basicstyle=\ttfamily\footnotesize,
%    breakatwhitespace=false,         
%    breaklines=true,                 
%    captionpos=b,                    
%    keepspaces=true,                 
%    numbers=none,                    
%    numbersep=5pt,                  
%    showspaces=false,                
%    showstringspaces=false,
%    showtabs=false,                  
%    tabsize=2
%}
%
%\lstdefinestyle{solution}{
%    backgroundcolor=\color{backcolour},   
%    commentstyle=\color{codegreen},
%    keywordstyle=\color{blue},
%    numberstyle=\tiny\color{codegray},
%    stringstyle=\color{codepurple},
%    basicstyle=\ttfamily\footnotesize,
%    breakatwhitespace=false,         
%    breaklines=true,                 
%    captionpos=b,                    
%    keepspaces=true,                 
%    numbers=left,                    
%    numbersep=5pt,                  
%    showspaces=false,                
%    showstringspaces=false,
%    showtabs=false,                  
%    tabsize=2
%}
%
%\lstset{style=task}
%\lstset{style=solution}


\newcommand{\reduction}[1]{\texttt{\textbf{[#1]}}}
\fancyhead[R]{27. Dezember 2022}

\setlength{\parindent}{0pt}

\begin{document}

\begin{center}
	\LARGE Reduktionsregeln
\end{center}

% ----------------------- TODO ---------------------------
% Hier werden die Aufgaben/Lösungen eingetragen:


\noindent Die Klammern $\llbracket\cdot\rrbracket^k,\ k \geq 0$ geben an wann
welcher Teilausdruck reduziert werden muss.

Der Ausdruck $\llbracket e\rrbracket^k$ mit dem \textbf{höchsten} $k$ wird
immer als nächstes reduziert!

\begin{contentbox}{Reduktionsregeln}
	\begin{enumerate}
		\item \textbf{Literal} $\ell$ (\texttt{1}, \texttt{\#t}, "\texttt{abc}", \texttt{\#<function:$\otimes$>}, ...): \hfill [\textbf{\texttt{eval\_lit}}]
		      \vspace{-1em}
		      \begin{align*}
			      \llbracket \ell \rrbracket^k \overset{\text{def}}{=} \ell
		      \end{align*}

		\item \textbf{Lambda-Abstraktion} \texttt{(\textbf{lambda} (v$_1$ ... v$_n$) e)}: \hfill \reduction{eval\_$\lambda$}
		      \vspace{-1em}
		      \begin{align*}
			      \llbracket \texttt{(\textbf{lambda} (v$_1$ ... v$_n$) e)} \rrbracket^k
			      \overset{\text{def}}{=} \texttt{(\textbf{lambda} (v$_1$ ... v$_n$) e)}
		      \end{align*}

		\item \textbf{Indentifier} id: \hfill \reduction{eval\_id}
		      \vspace{-1em}
		      \begin{align*}
			      \llbracket \text{id} \rrbracket^k
			      \overset{\text{def}}{=} \llbracket\texttt{e}\rrbracket^k,
		      \end{align*}
		      \vspace{-1em}
		      falls \texttt{(\textbf{define} id e)} gilt.

		\item \textbf{Applikation} \texttt{(f e$_1$ ... e$_n$)}:\\
		      Falls f = \texttt{\#<function:$\otimes$>} \hfill \reduction{apply\_prim}
		      \vspace{-1em}
		      \begin{align*}
			      \llbracket \texttt{(f e$_1$ ... e$_n$)} \rrbracket^k
			      \overset{\text{def}}{=}
			      \llbracket\llbracket e_1\rrbracket^{k+1} \otimes \dots \otimes \llbracket e_1\rrbracket^{k+1}\rrbracket^k
		      \end{align*}

		      \vspace{-1em}
		      Falls f = \texttt{(\textbf{lambda} (v$_1$ ... v$_n$) e)} \hfill \reduction{apply\_$\lambda$}
		      \vspace{-1em}
		      \begin{align*}
			      \llbracket \texttt{(f e$_1$ ... e$_n$)} \rrbracket^k
			      \overset{\text{def}}{=}
			      \llbracket
			      e\{v_1\to\llbracket e_1\rrbracket^{k+2},..., v_n\to\llbracket e_n\rrbracket^{k+2}\}^{k+1}
			      \rrbracket^k
		      \end{align*}

		      \vspace{-1em}
		      sonst \hfill \reduction{apply}
		      \vspace{-1em}
		      \begin{align*}
			      \llbracket \texttt{(f e$_1$ ... e$_n$)} \rrbracket^k
			      \overset{\text{def}}{=}
			      \llbracket(\llbracket \texttt{f}\rrbracket^{k+1} e_1 \dots e_n)\rrbracket^k
		      \end{align*}

		      \begin{errorbox}{Bitte beachten!}
			      Bei $\llbracket\text{\texttt{(\textbf{f} 1 2)}}\rrbracket^0$ weiß
			      man noch nicht, dass \texttt{\textbf{f}} eine $\lambda$-Abstraktion
			      ist. Man kann also \textbf{nicht} sofort
			      \texttt{\textbf{[apply\_$\lambda$]}} anwenden. Zuerst muss also
			      \reduction{apply} angewandt werden. Danach \reduction{eval\_id},
			      dann \reduction{eval\_$\lambda$}. Erst dann lässt sich
			      \reduction{apply\_$\lambda$} darauf werfen.
		      \end{errorbox}

		\item \textbf{Fallunterscheidung} \texttt{(\textbf{cond} (t$_1$ e$_1$) (t$_2$ e$_2$) ...)}:
		      \begin{itemize}
			      \item $\llbracket$
			            \texttt{(\textbf{cond} (else e$_1$))}
			            $\rrbracket^k \overset{\text{def}}{=}$
			            $\llbracket$ e$_1$
			            $\rrbracket^k$ \hfill \reduction{cond\_else}

			      \item $\llbracket$
			            \texttt{(\textbf{cond} (\#t e$_1$) (t$_2$ e$_2$) ...)}
			            $\rrbracket^k \overset{\text{def}}{=}$
			            $\llbracket$ e$_1$
			            $\rrbracket^k$ \hfill \reduction{cond\_\#t}

			      \item $\llbracket$
			            \texttt{(\textbf{cond} (\#f e$_1$) (t$_2$ e$_2$) ...)}
			            $\rrbracket^k \overset{\text{def}}{=}$
			            $\llbracket$
			            \texttt{(\textbf{cond} (t$_2$ e$_2$) ...)} $\rrbracket^k$
			            \hfill \reduction{cond\_\#f}

			      \item $\llbracket$
			            \texttt{(\textbf{cond})}$
				            \rrbracket^k \overset{\text{def}}{=}$
			            Abbruch $\lightning$ \hfill \reduction{cond\_fail}

			      \item $\llbracket$
			            \texttt{(\textbf{cond} (t$_1$ e$_1$) (t$_2$ e$_2$) ...)}
			            $\rrbracket^k \\\quad\overset{\text{def}}{=} \llbracket$
			            \texttt{(\textbf{cond} ($\llbracket t_1 \rrbracket^{k+1} e_1$) (t$_2$ e$_2$) ...)}
			            $\rrbracket^k$ \hfill \reduction{cond}
		      \end{itemize}
		      \begin{infobox}{\texttt{cond} ist "faul", d.h.}
			      \reduction{cond\_\#t} $t_2, e_2, ...$ werden nie ausgewertet.

			      \reduction{cond\_\#f} $e_1$ wird nie ausgewertet.
		      \end{infobox}
	\end{enumerate}
\end{contentbox}

\begin{contentbox}{Desugaring: \texttt{and/or -> if -> cond}}
	\begin{minipage}[t]{.5\linewidth}
		\texttt{(\textbf{and} t$_1$ t$_2$ ...) $\equiv$ (\textbf{if} t$_1$ (\textbf{and} t$_2$ ...) \#f)} \\
		\texttt{(\textbf{or} t$_1$ t$_2$ ...)~ $\equiv$ (\textbf{if} t$_1$ \#t (\textbf{or} t$_2$ ...))}
	\end{minipage}%
	\begin{minipage}[t]{.5\linewidth}
		\texttt{(\textbf{if} t$_1$ e$_1$ e$_2$)}\\
		\texttt{$\equiv$ (\textbf{cond} (t$_1$ e$_1$) (\textbf{else} e$_2$))}
	\end{minipage}%
\end{contentbox}


\newpage
\section*{Beispiel}

Wir definieren uns die Funktion \texttt{maybe-div}, die uns anstatt
\textit{/: division by zero} einfach \texttt{\#f} zurückgibt.

Links die "normale" Definition. Rechts die \textit{entzuckerte}, in der
\texttt{if} zu \texttt{cond} wird.

\begin{minipage}[t]{.5\linewidth}
	\begin{minted}[linenos]{racket}
(: maybe-div (number number 
              -> (mixed (enum #f) number)))
(define maybe-div
  (λ (x y)
    (if (= 0 y)
        #f
        (/ x y))))
\end{minted}
\end{minipage}%
\begin{minipage}[t]{.5\linewidth}
	\begin{minted}[linenos]{racket}
(: maybe-div (number number
              -> (mixed (enum #f) number)))
(define maybe-div
  (λ (x y)
    (cond ((= 0 y) #f)
          (else    (/ x y)))))
\end{minted}
\end{minipage}

\begin{center}
	\bfseries
	Reduktionsschritte
\end{center}

\footnotesize
\newcommand{\reduce}[2]{$\llbracket$\texttt{#1}$\rrbracket^#2$}
\newcommand{\accentcolor}{blue}
\newcommand{\highlighter}[1]{\textcolor{\accentcolor}{#1}}

\centering
\begin{minipage}{.8\linewidth}
	\footnotesize
	\noindent\texttt{\ \ }\reduce{{(\textbf{maybe-div} (* 4 21) 2)}}{0}\hfill\reduction{apply}

	\texttt{= }\reduce{(\reduce{{\textbf{maybe-div}}}{1} (* 4 21) 2)}{0}\hfill\reduction{eval\_id:maybe-div}

	\texttt{= }$\llbracket$($\llbracket$\texttt{(\textbf{lambda} (x y)}\hfill\reduction{eval\_$\lambda$}

	\quad\texttt{\ \ (\textbf{cond} ((\textbf{=} 0 y)\ \#f)}

	\quad\texttt{\ \ \ \ \ \ \ \ (\textbf{else}\ \ \ \ (\textbf{/} x y))))}$\rrbracket^1$ \texttt{(\textbf{*} 4 21) 2)}$\rrbracket^0$

	\texttt{= }$\llbracket$\texttt{((\textbf{lambda} (x y)}\hfill\reduction{apply\_$\lambda$}

	\quad\texttt{\ \ (\textbf{cond} ((\textbf{=} 0 y)\ \#f)}

	\quad\texttt{\ \ \ \ \ \ \ \ (\textbf{else}\ \ \ \ (\textbf{/} x y)))) (\textbf{*} 4 21) 2)}$\rrbracket^0$

	\texttt{= }$\llbracket$\texttt{(\textbf{cond} ((\textbf{=} 0 y)\ \#f)}\hfill\reduction{apply}

	\quad\texttt{\ \ \ \ \ (\textbf{else}\ \ \ \ (\textbf{/} x y)))\{x}$\to\llbracket$\texttt{(* 4 21)}$\rrbracket^2$, \texttt{y}$\to\llbracket$\texttt{2}$\rrbracket^2$\texttt{\}}$^1\rrbracket^0$

	\texttt{= }$\llbracket$\texttt{(\textbf{cond} ((\textbf{=} 0 y)\ \#f)}\hfill\reduction{eval\_id:*}


	\quad\texttt{\ \ \ \ \ (\textbf{else}\ \ \ \ (\textbf{/} x y)))\{x}$\to\llbracket$\texttt{(}$\llbracket$\texttt{\textbf{\#<function:*>}}$\rrbracket^3$\texttt{4 21)}$\rrbracket^2$, \texttt{y}$\to\llbracket$\texttt{2}$\rrbracket^2$\texttt{\}}$^1\rrbracket^0$

	\texttt{= }$\llbracket$\texttt{(\textbf{cond} ((\textbf{=} 0 y)\ \#f)}\hfill\reduction{eval\_lit:\#<function:*>}

	\quad\texttt{\ \ \ \ \ (\textbf{else}\ \ \ \ (\textbf{/} x y)))\{x}$\to\llbracket$\texttt{(}$\llbracket$\texttt{\textbf{\#<function:*>}}$\rrbracket^3$\texttt{4 21)}$\rrbracket^2$, \texttt{y}$\to\llbracket$\texttt{2}$\rrbracket^2$\texttt{\}}$^1\rrbracket^0$

	\texttt{= }$\llbracket$\texttt{(\textbf{cond} ((\textbf{=} 0 y)\ \#f)}\hfill\reduction{apply\_prim}

	\quad\texttt{\ \ \ \ \ (\textbf{else}\ \ \ \ (\textbf{/} x y)))\{x}$\to\llbracket$\texttt{(4 * 21)}$\rrbracket^2$, \texttt{y}$\to\llbracket$\texttt{2}$\rrbracket^2$\texttt{\}}$^1\rrbracket^0$

	\texttt{= }$\llbracket$\texttt{(\textbf{cond} ((\textbf{=} 0 y)\ \#f)}\hfill\reduction{eval\_lit:4,21}

	\quad\texttt{\ \ \ \ \ (\textbf{else}\ \ \ \ (\textbf{/} x y)))\{x}$\to\llbracket$\texttt{(\reduce{4}{3} * \reduce{21}{3})}$\rrbracket^2$, \texttt{y}$\to\llbracket$\texttt{2}$\rrbracket^2$\texttt{\}}$^1\rrbracket^0$

	\texttt{= }$\llbracket$\texttt{(\textbf{cond} ((\textbf{=} 0 y)\ \#f)}\hfill\reduction{multiply}

	\quad\texttt{\ \ \ \ \ (\textbf{else}\ \ \ \ (\textbf{/} x y)))\{x}$\to\llbracket$\texttt{(4 * 21)}$\rrbracket^2$, \texttt{y}$\to\llbracket$\texttt{2}$\rrbracket^2$\texttt{\}}$^1\rrbracket^0$

	\texttt{= }$\llbracket$\texttt{(\textbf{cond} ((\textbf{=} 0 y)\ \#f)}\hfill\reduction{eval\_lit:84,2}

	\quad\texttt{\ \ \ \ \ (\textbf{else}\ \ \ \ (\textbf{/} x y)))\{x}$\to\llbracket$\texttt{84}$\rrbracket^2$, \texttt{y}$\to\llbracket$\texttt{2}$\rrbracket^2$\texttt{\}}$^1\rrbracket^0$

	\texttt{= }$\llbracket$\texttt{(\textbf{cond} ((\textbf{=} 0 y)\ \#f)}\hfill\reduction{$\cdot\to\cdot$}

	\quad\texttt{\ \ \ \ \ (\textbf{else}\ \ \ \ (\textbf{/} x y)))\{x}$\to$\texttt{84, y}$\to$\texttt{2\}}$^1\rrbracket^0$

	\texttt{= }$\llbracket$\texttt{(\textbf{cond} ((\textbf{=} 0 2)\ \#f)}\hfill\reduction{cond}

	\quad\texttt{\ \ \ \ \ (\textbf{else}\ \ \ \ (\textbf{/} 84 2)))}$\rrbracket^0$

	\texttt{= }$\llbracket$\texttt{(\textbf{cond} (}$\llbracket$\texttt{(\textbf{=} 0 2)}$\rrbracket^1$\texttt{\ \#f)}\hfill\reduction{apply}

	\quad\texttt{\ \ \ \ \ (\textbf{else}\ \ \ \ (\textbf{/} 84 2)))}$\rrbracket^0$

	\texttt{= }$\llbracket$\texttt{(\textbf{cond} (}$\llbracket$\texttt{(}$\llbracket$\texttt{\textbf{=}}$\rrbracket^2$\texttt{ 0 2)}$\rrbracket^1$\texttt{\ \#f)}\hfill\reduction{eval\_id:=}

	\quad\texttt{\ \ \ \ \ (\textbf{else}\ \ \ \ (\textbf{/} 84 2)))}$\rrbracket^0$

	\texttt{= }$\llbracket$\texttt{(\textbf{cond} (}$\llbracket$\texttt{(}$\llbracket$\texttt{\textbf{\#<function:=>}}$\rrbracket^2$\texttt{ 0 2)}$\rrbracket^1$\texttt{\ \#f)}\hfill\reduction{eval\_lit:\#<function:=>}

	\quad\texttt{\ \ \ \ \ (\textbf{else}\ \ \ \ (\textbf{/} 84 2)))}$\rrbracket^0$

	\texttt{= }$\llbracket$\texttt{(\textbf{cond} (}$\llbracket$\texttt{0 = 2}$\rrbracket^1$\texttt{\ \#f)}\hfill\reduction{apply\_prim}

	\quad\texttt{\ \ \ \ \ (\textbf{else}\ \ \ \ (\textbf{/} 84 2)))}$\rrbracket^0$

	\texttt{= }$\llbracket$\texttt{(\textbf{cond} (\reduce{\reduce{0}{2} = \reduce{2}{2}}{1}\ \#f)}\hfill\reduction{eval\_lit:0,2}

	\quad\texttt{\ \ \ \ \ (\textbf{else}\ \ \ \ (\textbf{/} 84 2)))}$\rrbracket^0$

	\texttt{= }$\llbracket$\texttt{(\textbf{cond} (\reduce{0 = 2}{1}\ \#f)}\hfill\reduction{comparison}

	\quad\texttt{\ \ \ \ \ (\textbf{else}\ \ \ \ (\textbf{/} 84 2)))}$\rrbracket^0$

	\texttt{= }$\llbracket$\texttt{(\textbf{cond} (\reduce{\#f}{1}\ \#f)}\hfill\reduction{eval\_lit:\#f}

	\quad\texttt{\ \ \ \ \ (\textbf{else}\ \ \ \ (\textbf{/} 84 2)))}$\rrbracket^0$

	\texttt{= }$\llbracket$\texttt{(\textbf{cond} (\#f\ \#f)}\hfill\reduction{cond\_\#f}

	\quad\texttt{\ \ \ \ \ (\textbf{else}\ \ \ \ (\textbf{/} 84 2)))}$\rrbracket^0$

	\texttt{= }$\llbracket$\texttt{(\textbf{cond} (\textbf{else} (\textbf{/} 84 2)))}$\rrbracket^0$\hfill\reduction{cond\_else}

	\texttt{= \reduce{(\textbf{/} 84 2)}{0}}\hfill\reduction{apply}

	\texttt{= \reduce{(\reduce{/}{1} 84 2}{0}}\hfill\reduction{eval\_id:/}

	\texttt{= \reduce{(\reduce{\textbf{\#<function:/>}}{1} 84 2}{0}}\hfill\reduction{eval\_lit:\#<function:/>}

	\texttt{= \reduce{84 / 2}{0}}\hfill\reduction{apply\_prim}

	\texttt{= \reduce{\reduce{84}{1} / \reduce{2}{1}}{0}}\hfill\reduction{eval\_lit:84:2}

	\texttt{= \reduce{84 / 2}{1}}\hfill\reduction{Division}

	\texttt{= \reduce{42}{1}}\hfill\reduction{eval\_lit:42}

	\texttt{= \underline{42}}
\end{minipage}


\end{document}
